%%%%%%%%%%%%%%%%%%%%%%%%%%%%%%%%%%%%%%%%%%%

\documentclass[11pt, a4paper, twoside]{article}
\usepackage[utf8]{inputenc}
\usepackage[T1]{fontenc}
\usepackage[francais]{babel}
\usepackage{lmodern}
\usepackage[margin=0.8in]{geometry}

\usepackage{amsmath}
\usepackage{amssymb}
\usepackage{amsthm}
\usepackage{mathrsfs}
\usepackage{mathtools}
\usepackage{enumitem}
\usepackage{graphicx}
\usepackage[all]{xy}

\DeclarePairedDelimiter\ceil{\lceil}{\rceil}
\DeclarePairedDelimiter\floor{\lfloor}{\rfloor}

% environments
\newtheorem{thm}{Theorem}[subsection]
\newtheorem{lemma}[thm]{Lemma}
\newtheorem{prop}[thm]{Proposition}
\newtheorem{cor}[thm]{Corollary}
\newtheorem{conj}[thm]{Conjecture}
\newtheorem{thmi}{Theorem}     % separate numbering for the introduction
\newtheorem{conji}{Conjecture} % separate numbering for the introduction
\newtheorem{claimi}{Claim} % separate numbering

\theoremstyle{definition} 
\newtheorem{defin}[thm]{Definition}
\newtheorem{cons}[thm]{Construction}
\newtheorem{remark}[thm]{Remark}
\newtheorem{variant}[thm]{Variant}
\newtheorem{example}[thm]{Example}

% script letters
\newcommand{\cA}{\mathcal{A}}
\newcommand{\cB}{\mathscr B}
\newcommand{\cC}{\mathcal{C}}
\newcommand{\cD}{\mathscr D}
\newcommand{\cE}{\mathscr E}
\newcommand{\cF}{\mathscr F}
\newcommand{\cG}{\mathscr G}
\newcommand{\cH}{\mathscr H}
\newcommand{\cI}{\mathscr I}
\newcommand{\cJ}{\mathscr J}
\newcommand{\cK}{\mathscr K}
\newcommand{\cL}{\mathscr L}
\newcommand{\cM}{\mathscr M}
\newcommand{\cN}{\mathscr N}
\newcommand{\cO}{\mathscr O}
\newcommand{\cP}{\mathscr P}
\newcommand{\cR}{\mathcal{R}}
\newcommand{\cS}{\mathscr S}
\newcommand{\cQ}{\mathscr Q}
\newcommand{\cT}{\mathscr T}
\newcommand{\cU}{\mathscr U}
\newcommand{\cW}{\mathscr W}
\newcommand{\cX}{\mathscr X}
\newcommand{\cY}{\mathscr Y}
\newcommand{\cZ}{\mathscr Z}

% boldface letters
\newcommand{\bb}[1]{\mathbf{#1}} 
\newcommand{\FF}{\bb{F}}
\newcommand{\GG}{\bb{G}}
\newcommand{\NN}{\bb{N}}
\newcommand{\PP}{\bb{P}}
\newcommand{\QQ}{\bb{Q}}
\newcommand{\RR}{\bb{R}}
\newcommand{\ZZ}{\bb{Z}}
\newcommand{\bt}{\mathbf}

%raccourcis
\newcommand{\depth}{\mathrm{depth}}
\newcommand{\height}{\mathrm{ht}}
\newcommand{\Id}{\mathrm{Id}}
\newcommand{\Mcomp}{M_{\bullet}}
\newcommand{\Ncomp}{N_{\bullet}}
\newcommand{\A}{\mathbb{A}}
\newcommand{\C}{\mathbb{C}}
\newcommand{\Z}{\mathbb{Z}}
\newcommand{\Q}{\mathbb{Q}}
\newcommand{\N}{\mathbb{N}}
\newcommand{\K}{\mathbb{K}}
\newcommand{\cechC}{\mathrm{\textit{\v{C}}}}
\newcommand{\cechH}{\mathrm{\textit{\v{H}}}}
\newcommand{\injdim}{\mathrm{injdim}}
\newcommand{\pd}{\mathrm{pd}}
\newcommand{\aideal}{\mathfrak{a}}
\newcommand{\mideal}{\mathfrak{m}}
\newcommand{\nideal}{\mathfrak{n}}
\newcommand{\pideal}{\mathfrak{p}}
\newcommand{\qideal}{\mathfrak{q}}
\newcommand{\rank}{\mathrm{rank}}


% symbols
\renewcommand{\phi}{\varphi}
\newcommand{\isom}{\simeq} 
\newcommand{\piet}{\pi^{\acute{e}t}_{1}}
\newcommand{\et}{{\rm\acute{e}t}}
\newcommand{\cat}[1]{{\normalfont\textbf{#1}}}  % categories
\newcommand{\ra}{\longrightarrow}    % to be used instead of \to in displayed formulas
\newcommand{\isomto}{\xrightarrow{\,\smash{\raisebox{-0.65ex}{\ensuremath{\scriptstyle\sim}}}\,}}
\renewcommand{\bigwedge}{\mbox{\large $\wedge$}}


% operators
\DeclareMathOperator{\Alb}{Alb}
\DeclareMathOperator{\Art}{\cat{Art}} 
\DeclareMathOperator{\Aut}{Aut}
\DeclareMathOperator{\Bl}{Bl}
\DeclareMathOperator{\Def}{Def}
\DeclareMathOperator{\Ext}{Ext}
\DeclareMathOperator{\Gr}{Gr}

%basic stuff
\newcommand{\tens}[1]{%
  \mathbin{\mathop{\otimes}\displaylimits_{#1}}%
}
\DeclareMathOperator{\Hom}{Hom}
\DeclareMathOperator{\Pic}{Pic}
\DeclareMathOperator{\Proj}{Proj}
\DeclareMathOperator{\Sing}{Sing}
\DeclareMathOperator{\Spec}{Spec}
\DeclareMathOperator{\Sym}{Sym}
\DeclareMathOperator{\Tor}{Tor}
\DeclareMathOperator{\Ass}{Ass}
\DeclareMathOperator{\Ann}{Ann}
\DeclareMathOperator{\Nil}{Nil}
\DeclareMathOperator{\Supp}{Supp}
\DeclareMathOperator{\Frac}{Frac}
\DeclareMathOperator{\Ob}{Ob}
\DeclareMathOperator{\Mor}{Mor}
\DeclareMathOperator{\Fun}{Fun}
\DeclareMathOperator{\Nat}{Nat}
% sheaf Hom etc
\newcommand{\cExt}{{\mathscr E}\kern -.5pt xt}
\newcommand{\cHom}{\mathscr{H}\kern -.5pt om}
\newcommand{\cEnd}{{\mathscr E}nd}

% other
\newcommand{\stacksproj}[1]{{\cite[Tag~\href{http://stacks.math.columbia.edu/tag/#1}{#1}]{stacks-project}}}

% tildes
% - for uppercase letters X Y Z F D U R A ... use \wt X etc
% - for lowercase letters \pi ... use \swt \pi 
% - with the exception of "f" and "s" (used in one proof), where I used \tilde
% - for script I, R (ideal used in section 3.4) use \wtcI, \wtcR
% - for complex expressions e.g. X\times Y use \widetilde{...}

\newcommand{\wt}[1]{{\mathchoice%
  {\raisebox{1.5ex}{\resizebox{1.7ex}{!}{{}\hphantom{i}\ensuremath{{\sim}}}} \hspace{-1.7ex}{#1}}%
  {\smash{\raisebox{1.5ex}{\resizebox{1.7ex}{!}{{}\hphantom{i}\ensuremath{{\sim}}}}\hspace{-1.7ex}{#1 }}\vphantom{\tilde I}}%
  {\raisebox{1.1ex}{\resizebox{1.3ex}{!}{{}\hphantom{i}\ensuremath{{\sim}}}}\hspace{-1.3ex}{#1}}%
  {\raisebox{0.8ex}{\resizebox{1ex}{!}{{}\hphantom{i}\ensuremath{{\sim}}}}\hspace{-1ex}{#1}}%
}}

\newcommand{\swt}[1]{{\mathchoice%
  {\raisebox{0.9ex}{\resizebox{1.2ex}{!}{\ensuremath{{\sim}}}}\hspace{-1.4ex}{#1}}%
  {\smash{\raisebox{0.9ex}{\resizebox{1.2ex}{!}{\ensuremath{{\sim}}}}\hspace{-1.4ex}{#1 }}\vphantom{I}}%
  {\raisebox{0.7ex}{\resizebox{0.8ex}{!}{\ensuremath{{\sim}}}}\hspace{-0.9ex}{#1}}%
  {\raisebox{0.5ex}{\resizebox{1ex}{!}{{}\hphantom{i}\ensuremath{{\sim}}}}\hspace{-1ex}{#1}}%
}}

\newcommand{\wtcI}{\smash{\raisebox{1.5ex}{\hspace{0.7ex}\resizebox{1.2ex}{!}{\ensuremath{{\sim}}}}\hspace{-2.1ex}{\cI}}\vphantom{I}}

\newcommand{\wtcR}{\smash{\raisebox{1.5ex}{\hspace{0.7ex}\resizebox{1.2ex}{!}{\ensuremath{{\sim}}}}\hspace{-2.1ex}{\cR}}\vphantom{I}}

%%%%%%%%%%%%%%%%%%%%%%%%%%%%%%%%%%%%%%%%%%%


\begin{document}
\title{Modern Algebraic Geometry\\ Hand in 2}
\author{Christophe Marciot}
\maketitle

\section*{Part (a):}
We have to verify that $I_Y$ checks the gluing properties of a sheaf. If $s\in I_Y(U)$ and $\{V_i\}_{i\in I}$ is an open cover of $U$ sucht that $s\restr{V_i}=0$ for each $i$ then it is clear that $s=0$ as this property comes from the fact that $\sO_X$ is a sheaf. Now if we have $s_i\in I_Y(V_i)$, where $\{V_i\}_{i\in I}$ is again an open cover of $U$ such that $s_i\restr{V_{ij}}=s_j\restr{V_{ij}}$, we know we can glue back those $s_i$'s to some $s\in\sO_X(U)$. To see that $s$ is in fact in $I_Y(U)$, we observe that for $p\in U$, in particular $p\in V_i$ for some $i\in I$, we have 
\begin{displaymath}
	s(p)=s\restr{V_i}(p)=s_i(p)=0
\end{displaymath} 
since $s_i$ vanishes on $U\cap V_i$ and thus $s\in I_Y(U)$. We then conclude that $I_Y$ is a sheaf.

\section*{Part (b):}
Here we first begin by a remark. Note that for $f\in\sO_X(U)$, then for $p\in U$, one can associate $f_p$ with $f(p)$. Indeed, suppose that for $g\in\sO_X(V),p\in V$ such that $f_p=g_p$. Then we get that there exists a neighbourhood of $p$,$W\subseteq U\cap V$ such that $f\restr{W}=g\restr{W}$ and thus 
\begin{displaymath}
	f(p)=f\restr{W}(p)=g\restr{W}(p)=g(p).
\end{displaymath}
Now if $f(p)=g(p)$ we get that $(f-g)(p)=0$. Since $\Supp(f-g)$ is an open set, we know there exists a neighbourhood $U\subseteq\Supp(f-g)$ of $p$. This said, we get that $f\restr{U}=g\restr{U}$ and thus $f_p=g_p$.\\
Now let us go back to the problem at hand. Let us define the morphism of sheaves 
\begin{displaymath}
	\phi_U:\sO_X(U)/I_Y(U)\ra(\iota_*\sO_Y)(U)=\sO(Y\cap U):\bar{f}\ra f\restr{Y\cap U},
\end{displaymath}
where $\bar{f}$ is the class of $f$ in $\sO_X(U)/I_Y(U)$. Note that if $\bar{f}=\bar{g}$, then $f=g+r$ with $r\in I_Y(U)$. Thus
\begin{displaymath}
	f\restr{Y\cap U}=(g+r)\restr{Y\cap U}=g\restr{Y\cap U}+r\restr{Y\cap U} = g\restr{Y\cap U}
\end{displaymath}
and thus $\phi_U$ is a well defined map. Note that these are homomorphisms of rings. To show that this is a morphism of sheaves, we go to the level of stlaks. Let $p\in X$. we distinguish two cases. If $p\notin Y$, then we have 
\begin{displaymath}
	I_{Y,p}= \lim\limits_{\ra p\in U}I_Y(U).
\end{displaymath}
Note that since $p\notin \bar{Y}=Y$, we know there exists a neighbourhood of $p$,$U$, such that $Y\cap U=\emptyset$. We then get that for each $p\in V\subseteq U$ open, one has $I_Y(V)=\sO_X(V)$ and thus, by te properties of the direct limit, we have that $I_{Y,p}=\sO_{X,p}$ and thus 
\begin{displaymath}
	(\sO_X/I_Y)_p=\sO_{X,p}/I_{Y,p}=\sO_{X,p}/\sO_{X,p}=0.
\end{displaymath}
Also note that for small enough neighbourhoods $U$ of $p$, we have
\begin{displaymath}
	(\iota_*\sO_Y)(U)=\sO_Y(Y\cap U)=\sO_Y(\emptyset)=0
\end{displaymath}
and thus
\begin{displaymath} 
	(\iota_*\sO_Y)_p=\lim\limits_{\ra p\in U}(\iota_*\sO_Y)(U)=\lim\limits_{\ra p\in U}\sO_Y(Y\cap U)=0.
\end{displaymath}
We then deduce that $\phi_p$ is an isomorphism. Now suppose that $p\in Y$. Since $p\in Y$, we know that $f(p)=0$  for all neighbourhoods $U$ of $p$ and $f\in I_Y(U)$ and thus we conclude that $I_{Y,p}=0$. Thus $\phi_p$ becomes
\begin{displaymath}
	\phi_p:\sO_{X,p}\ra\sO_{Y,p}:[(U,f)]\ra[(Y\cap U,f\restr{Y\cap U})]
\end{displaymath}
and thus $\phi_p$ is an isomorphism. This tells us that $\phi$ is an isomorphism.
	
	
\section*{Part (c):}
Let us consider $Y=\{p,q\}$. Recall that singletons are closed in $X$ and thus $Y$ is closed. We want to show that $\iota_*\sO_Y\isom\iota_*\sO_p\oplus\iota_*\sO_q$. We set 
\begin{displaymath}
	\phi_U:\sO_Y(Y\cap U)\ra\sO_p(\{p\}\cap U)\oplus\sO_q(\{q\}\cap U):f\ra(f\restr{\{p\}},f\restr{\{q\}}),
\end{displaymath}
By noting that $\iota_* \sO_{Y,x},\iota_* \sO_{p,x},\iota_* \sO_{q,x}=0$ if $x\neq p,q$, and $\iota_* \sO_{Y,p}=\iota_* \sO_{p,p}$ and $\iota_* \sO_{q,p}=0$, and $\iota_* \sO_{Y,q}=\iota_* \sO_{q,q}$ and $\iota_* \sO_{p,q}=0$, we see that $\phi$ is an isomorphism of sheaves. Note that $\phi_U$ and $\phi_p$ are ring homomorphisms for each $U\subseteq X$ open and $p\in X$. Also we note that $\sO_X(X)\isom k$. where $k$ is the undelying field (since $X$ is a projective variety) and $\cF(X)\isom k\oplus k$ and thus the morphism $\phi_X:\Gamma(X,\sO_X)\ra\Gamma(X,\cF)$ can not reach the zerodivisors of $k\oplus k$ (elements like $(1,0)$ for example) and we conclude that $\phi_X$ is not surjective.










\end{document}
