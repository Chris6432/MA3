%%%%%%%%%%%%%%%%%%%%%%%%%%%%%%%%%%%%%%%%%%%

\documentclass[11pt, a4paper, twoside]{article}
\usepackage[utf8]{inputenc}
\usepackage[T1]{fontenc}
\usepackage[francais]{babel}
\usepackage{lmodern}
\usepackage[margin=0.8in]{geometry}

\usepackage{amsmath}
\usepackage{amssymb}
\usepackage{amsthm}
\usepackage{mathrsfs}
\usepackage{mathtools}
\usepackage{enumitem}
\usepackage{graphicx}
\usepackage[all]{xy}

\DeclarePairedDelimiter\ceil{\lceil}{\rceil}
\DeclarePairedDelimiter\floor{\lfloor}{\rfloor}

% environments
\newtheorem{thm}{Theorem}[subsection]
\newtheorem{lemma}[thm]{Lemma}
\newtheorem{prop}[thm]{Proposition}
\newtheorem{cor}[thm]{Corollary}
\newtheorem{conj}[thm]{Conjecture}
\newtheorem{thmi}{Theorem}     % separate numbering for the introduction
\newtheorem{conji}{Conjecture} % separate numbering for the introduction
\newtheorem{claimi}{Claim} % separate numbering

\theoremstyle{definition} 
\newtheorem{defin}[thm]{Definition}
\newtheorem{cons}[thm]{Construction}
\newtheorem{remark}[thm]{Remark}
\newtheorem{variant}[thm]{Variant}
\newtheorem{example}[thm]{Example}

% script letters
\newcommand{\cA}{\mathcal{A}}
\newcommand{\cB}{\mathscr B}
\newcommand{\cC}{\mathcal{C}}
\newcommand{\cD}{\mathscr D}
\newcommand{\cE}{\mathscr E}
\newcommand{\cF}{\mathscr F}
\newcommand{\cG}{\mathscr G}
\newcommand{\cH}{\mathscr H}
\newcommand{\cI}{\mathscr I}
\newcommand{\cJ}{\mathscr J}
\newcommand{\cK}{\mathscr K}
\newcommand{\cL}{\mathscr L}
\newcommand{\cM}{\mathscr M}
\newcommand{\cN}{\mathscr N}
\newcommand{\cO}{\mathscr O}
\newcommand{\cP}{\mathscr P}
\newcommand{\cR}{\mathcal{R}}
\newcommand{\cS}{\mathscr S}
\newcommand{\cQ}{\mathscr Q}
\newcommand{\cT}{\mathscr T}
\newcommand{\cU}{\mathscr U}
\newcommand{\cW}{\mathscr W}
\newcommand{\cX}{\mathscr X}
\newcommand{\cY}{\mathscr Y}
\newcommand{\cZ}{\mathscr Z}

% boldface letters
\newcommand{\bb}[1]{\mathbf{#1}} 
\newcommand{\FF}{\bb{F}}
\newcommand{\GG}{\bb{G}}
\newcommand{\NN}{\bb{N}}
\newcommand{\PP}{\bb{P}}
\newcommand{\QQ}{\bb{Q}}
\newcommand{\RR}{\bb{R}}
\newcommand{\ZZ}{\bb{Z}}
\newcommand{\bt}{\mathbf}

%raccourcis
\newcommand{\depth}{\mathrm{depth}}
\newcommand{\height}{\mathrm{ht}}
\newcommand{\Id}{\mathrm{Id}}
\newcommand{\Mcomp}{M_{\bullet}}
\newcommand{\Ncomp}{N_{\bullet}}
\newcommand{\A}{\mathbb{A}}
\newcommand{\C}{\mathbb{C}}
\newcommand{\Z}{\mathbb{Z}}
\newcommand{\Q}{\mathbb{Q}}
\newcommand{\N}{\mathbb{N}}
\newcommand{\K}{\mathbb{K}}
\newcommand{\cechC}{\mathrm{\textit{\v{C}}}}
\newcommand{\cechH}{\mathrm{\textit{\v{H}}}}
\newcommand{\injdim}{\mathrm{injdim}}
\newcommand{\pd}{\mathrm{pd}}
\newcommand{\aideal}{\mathfrak{a}}
\newcommand{\mideal}{\mathfrak{m}}
\newcommand{\nideal}{\mathfrak{n}}
\newcommand{\pideal}{\mathfrak{p}}
\newcommand{\qideal}{\mathfrak{q}}
\newcommand{\rank}{\mathrm{rank}}


% symbols
\renewcommand{\phi}{\varphi}
\newcommand{\isom}{\simeq} 
\newcommand{\piet}{\pi^{\acute{e}t}_{1}}
\newcommand{\et}{{\rm\acute{e}t}}
\newcommand{\cat}[1]{{\normalfont\textbf{#1}}}  % categories
\newcommand{\ra}{\longrightarrow}    % to be used instead of \to in displayed formulas
\newcommand{\isomto}{\xrightarrow{\,\smash{\raisebox{-0.65ex}{\ensuremath{\scriptstyle\sim}}}\,}}
\renewcommand{\bigwedge}{\mbox{\large $\wedge$}}


% operators
\DeclareMathOperator{\Alb}{Alb}
\DeclareMathOperator{\Art}{\cat{Art}} 
\DeclareMathOperator{\Aut}{Aut}
\DeclareMathOperator{\Bl}{Bl}
\DeclareMathOperator{\Def}{Def}
\DeclareMathOperator{\Ext}{Ext}
\DeclareMathOperator{\Gr}{Gr}

%basic stuff
\newcommand{\tens}[1]{%
  \mathbin{\mathop{\otimes}\displaylimits_{#1}}%
}
\DeclareMathOperator{\Hom}{Hom}
\DeclareMathOperator{\Pic}{Pic}
\DeclareMathOperator{\Proj}{Proj}
\DeclareMathOperator{\Sing}{Sing}
\DeclareMathOperator{\Spec}{Spec}
\DeclareMathOperator{\Sym}{Sym}
\DeclareMathOperator{\Tor}{Tor}
\DeclareMathOperator{\Ass}{Ass}
\DeclareMathOperator{\Ann}{Ann}
\DeclareMathOperator{\Nil}{Nil}
\DeclareMathOperator{\Supp}{Supp}
\DeclareMathOperator{\Frac}{Frac}
\DeclareMathOperator{\Ob}{Ob}
\DeclareMathOperator{\Mor}{Mor}
\DeclareMathOperator{\Fun}{Fun}
\DeclareMathOperator{\Nat}{Nat}
% sheaf Hom etc
\newcommand{\cExt}{{\mathscr E}\kern -.5pt xt}
\newcommand{\cHom}{\mathscr{H}\kern -.5pt om}
\newcommand{\cEnd}{{\mathscr E}nd}

% other
\newcommand{\stacksproj}[1]{{\cite[Tag~\href{http://stacks.math.columbia.edu/tag/#1}{#1}]{stacks-project}}}

% tildes
% - for uppercase letters X Y Z F D U R A ... use \wt X etc
% - for lowercase letters \pi ... use \swt \pi 
% - with the exception of "f" and "s" (used in one proof), where I used \tilde
% - for script I, R (ideal used in section 3.4) use \wtcI, \wtcR
% - for complex expressions e.g. X\times Y use \widetilde{...}

\newcommand{\wt}[1]{{\mathchoice%
  {\raisebox{1.5ex}{\resizebox{1.7ex}{!}{{}\hphantom{i}\ensuremath{{\sim}}}} \hspace{-1.7ex}{#1}}%
  {\smash{\raisebox{1.5ex}{\resizebox{1.7ex}{!}{{}\hphantom{i}\ensuremath{{\sim}}}}\hspace{-1.7ex}{#1 }}\vphantom{\tilde I}}%
  {\raisebox{1.1ex}{\resizebox{1.3ex}{!}{{}\hphantom{i}\ensuremath{{\sim}}}}\hspace{-1.3ex}{#1}}%
  {\raisebox{0.8ex}{\resizebox{1ex}{!}{{}\hphantom{i}\ensuremath{{\sim}}}}\hspace{-1ex}{#1}}%
}}

\newcommand{\swt}[1]{{\mathchoice%
  {\raisebox{0.9ex}{\resizebox{1.2ex}{!}{\ensuremath{{\sim}}}}\hspace{-1.4ex}{#1}}%
  {\smash{\raisebox{0.9ex}{\resizebox{1.2ex}{!}{\ensuremath{{\sim}}}}\hspace{-1.4ex}{#1 }}\vphantom{I}}%
  {\raisebox{0.7ex}{\resizebox{0.8ex}{!}{\ensuremath{{\sim}}}}\hspace{-0.9ex}{#1}}%
  {\raisebox{0.5ex}{\resizebox{1ex}{!}{{}\hphantom{i}\ensuremath{{\sim}}}}\hspace{-1ex}{#1}}%
}}

\newcommand{\wtcI}{\smash{\raisebox{1.5ex}{\hspace{0.7ex}\resizebox{1.2ex}{!}{\ensuremath{{\sim}}}}\hspace{-2.1ex}{\cI}}\vphantom{I}}

\newcommand{\wtcR}{\smash{\raisebox{1.5ex}{\hspace{0.7ex}\resizebox{1.2ex}{!}{\ensuremath{{\sim}}}}\hspace{-2.1ex}{\cR}}\vphantom{I}}

%%%%%%%%%%%%%%%%%%%%%%%%%%%%%%%%%%%%%%%%%%%


\begin{document}
\title{Hand in 1}
\author{Christophe Marciot}
\maketitle

\section{Introduction}
	In this document, we will consider sheaves on a fixed topological space $X$ if not precised otherwise.

\subsection{Notation}
	Let $\cF$ be a sheaf. For $p\in X$, we will use two different notations for the elements of $\cF_p$. The first one is $s_p$. The second one is $[(s,U)]$. This means that we have an neighbourhood of $p$, $U$, and an element $s\in\cF(U)$. Note that in the first notation, the existence of such a $U$ is implicit. That is why we will use the second notation when we need to explicitly describe $U$. Also we sometimes will write $[(s,U)]_p$ when it is not clear in what stalk we work.

\subsection{Preliminary result}
	\begin{prop}
		Let $\phi:\cF\ra\cG$ be a morphism of sheaves. Then for each $s\in\cF(U)$ and $p\in U$, we have 
			\begin{displaymath}
				(\phi_U(s))_p=\phi_p(s_p).
			\end{displaymath}
	\end{prop}
	\begin{proof}
		Note that $s_p=[(s,U)]_p$ and so $\phi_p(s_p)=[(\phi_U(s),U)]$ and $(\phi_U(s))_p=[(\phi_U(s),U)]_p$ by definition.
	\end{proof}




\section{Homeworks}

\subsection{Exercise 1.2., part (a)}
	First, let us look at the kernel part. Let $[(s,U)]\in\Ker(\phi_p)$. Then we know that $[(\phi_U(s),U)]=0$ in $\cG_p$. in other words, we get that there exists $V_p\subset U$ neighbourhood of $p$ such that $\phi_U(s)\restr{V_p}=0$ in $\cG(V_p)$. Note that since $\phi$ is a morphism, we have that 
	\begin{displaymath}
		\phi_{V_p}(s\restr{V_p})=\phi_U(s)\restr{V_p}=0
	\end{displaymath}
	and thus $s\restr{V_p}\in(\Ker\phi)(V_p)$. We then get that $[(s\restr{V_p},V_p)]=[(s,U)]\in(\Ker\phi)_p$. We then 		conclude that $\Ker(\phi_p)\subset(\Ker\phi)_p$. Now let $[(s,U)]\in(\Ker\phi)_p$. Note that $s\in(\Ker\phi)(U)=\Ker\phi_U$. We then observe that 
	\begin{displaymath}
		\phi_p(s_p)=(\phi_U(s))_p=0\ \mathrm{in}\ \cG_p
	\end{displaymath}
	adn thus $[(s,U)]\in\Ker\phi_p$. This allows us to conclude that $\Ker\phi_p=(\Ker\phi)_p$.\\
	Now let us look at the image sheaf. One has 
	\begin{align*}
		[(t,V)]\in\im\phi_p &\iff \exists [(s,U)]\in\cF_p\ \mathrm{s.th.}\ \phi_p(s_p)=t_p\\
						  &\iff \exists [(s,U)]\in\cF_p\ \mathrm{s.th.}\ [(\phi_U(s),U)]=[(t,V)]\\
						  &\iff \exists [(s,U)]\in\cF_p\ \mathrm{s.th.}\ \exists W\subset U\cap V\ \mathrm{a\ neighbourhood\ of}\ p\ \mathrm{s.th.}\ \phi_U(s)\restr{W}=t\restr{W}\\
						  &\iff \exists [(s,U)]\in\cF_p\ \mathrm{s.th.}\ \exists W\subset U\cap V\ \mathrm{a\ neighbourhood\ of}\ p\ \mathrm{s.th.}\ \phi_W(s\restr{W})=t\restr{W}\\
						  &\iff \exists W\subset V\ \mathrm{a\ neighbourhood\ of}\ p\ \mathrm{s.th.}\ t\restr{W}\in\im\phi_W\\
						  &\iff \exists W\subset V\ \mathrm{a\ neighbourhood\ of}\ p\ \mathrm{s.th.}\ [(t,V)]=[(t,W)]\in(\im\phi)_p\\
	\end{align*}

\subsection{Exercise 1.2., part (b)}
	First let us investigate the injectivity. We recall that $\phi$ is injective if the kernel sheaf is trivial. Suppose that $\phi$ is injective. As $\Ker\phi=0$, we have 
	\begin{displaymath}
		\Ker\phi_p=(\Ker\phi)_p=0_p=0,\forall p\in X
	\end{displaymath}
	and so $\phi_p$ is injective for all $p\in X$. Now suppose that $\phi_p$ is injective for all $p\in X$. Let $U\subset X$ be an open set and let $s\in\Ker\phi(U)$. Then $s_p\in(\Ker\phi)_p=\Ker\phi_p$ for each $p\in U$ and thus $s_p=0$ for each $p\in U$. Note that this means that
	\begin{displaymath}
		\forall p\in U\exists V_p\subset U\ \mathrm{a\ neighbourhood\ of}\ p\ \mathrm{s.th.}\ s\restr{V_p}=0.
	\end{displaymath}
	Since the family $\{V_p\}_{p\in U}$ is an open cover of $U$ and $\cF$ is a sheaf, we get that $s=0$. We then conclude that the assertion holds.\\
	Next let us investigate surjectivity. Recall that $\phi$ is surjective if $\im\phi=\cG$. Suppose that $\phi$ is surjective. Then 
	\begin{displaymath}
		\im\phi_p=(\im\phi)_p=\cG_p,\forall p\in X
	\end{displaymath}
	and thus $\phi_p$ is surjective for each $p\in X$. Now suppose that $\phi_p$ is surjective for each $p\in X$. Let us consider the inclusion morphism $\iota:\im\phi\ra\cG$. This morphisms at a given stalk is the inclusion map of the subgroup $(\im\phi)_p$ into the group $\cG_p$. Since $(\im\phi)_p=\im\phi_p=\cG_p$, because of the surjectivity of $\phi_p$, we get that $\iota_p$ is an isomorphism for each $p\in X$. Thus we get that $\iota$ is an isomorphism which means that $\im\phi=\cG$.
	
\subsection{Exercise 1.2., part (c)}
	We have that 
	\begin{align*}
		\xymatrix{\cdots\ar[r]\ & \cF^{i-1} \ar[r]^{\phi^{i-1}} & \cF^{i} \ar[r]^{\phi^{i}} & \cF^{i+1} \ar[r] & \cdots}\ \mathrm{is\ an\ exact\ sequence\ of\ sheaves}\\ \iff \Ker\phi^{i}=\im\phi^{i-1},\forall i\in\Z\\
												   \iff \phi^{i-1}:\cF^{i-1}\ra\Ker\phi^{i}\ \mathrm{is\ surjective},\forall i\in\Z\\
												   \iff (\phi^{i-1})_p:(\cF^{i-1})_p\ra(\Ker\phi^{i})_p\ \mathrm{is\ surjective},\forall p\in X,\forall i\in\Z\\
												   \iff (\im\phi^{i-1})_p=(\Ker\phi^{i})_p,\forall p\in X,\forall i\in\Z\\
												   \iff \im\phi^{i-1}_p=\Ker\phi^{i}_p,\forall p\in X,\forall i\in\Z\\ 
												   \iff \xymatrix{\cdots\ar[r]\ & \cF^{i-1}_p \ar[r]^{\phi^{i-1}_p} & \cF^{i}_p \ar[r]^{\phi^{i}_p} & \cF^{i+1}_p \ar[r] & \cdots}\ \mathrm{is\ an\ exact\ sequence\ of\ groups},\forall p\in X
	\end{align*}







\end{document}