%%%%%%%%%%%%%%%%%%%%%%%%%%%%%%%%%%%%%%%%%%%

\documentclass[11pt, a4paper, twoside]{article}
\usepackage[utf8]{inputenc}
\usepackage[T1]{fontenc}
\usepackage[francais]{babel}
\usepackage{lmodern}

\usepackage{amsmath}
\usepackage{amssymb}
\usepackage{amsthm}
\usepackage{mathrsfs}

%%%%%%%%%%%%%%%%%%%%%%%%%%%%%%%%%%%%%%%%%%%


\begin{document}
\title{Exercise sheet 1 - Exercise 6}
\author{Christophe Marciot}
\maketitle

Suppose we have a natural transformation $\tau:F\rightarrow G$. We want to associate to $\tau$ a functor $H_\tau:\mathscr{C}\times[1]\rightarrow\mathscr{D}$ in a meaningful way. In fact we want that this association to remind us of the notion of homotopy in regular homotopy theory. \\
We propose the follwing definition: For an object $(c,x)\in \mathrm{Ob}\mathscr{C}\times[1]$ and an arrow $(f,p)\in \mathrm{Ob}\mathscr{C}\times[1]((c,x),(c',x'))$.
	\begin{displaymath}
	\end{displaymath}
		



\end{document}
