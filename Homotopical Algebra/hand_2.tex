%%%%%%%%%%%%%%%%%%%%%%%%%%%%%%%%%%%%%%%%%%%

\documentclass[11pt, a4paper, twoside]{article}
\usepackage[utf8]{inputenc}
\usepackage[T1]{fontenc}
\usepackage[francais]{babel}
\usepackage{lmodern}
\usepackage[margin=0.8in]{geometry}

\usepackage{amsmath}
\usepackage{amssymb}
\usepackage{amsthm}
\usepackage{mathrsfs}
\usepackage{mathtools}
\usepackage{enumitem}
\usepackage{graphicx}
\usepackage[all]{xy}

\DeclarePairedDelimiter\ceil{\lceil}{\rceil}
\DeclarePairedDelimiter\floor{\lfloor}{\rfloor}

% environments
\newtheorem{thm}{Theorem}[subsection]
\newtheorem{lemma}[thm]{Lemma}
\newtheorem{prop}[thm]{Proposition}
\newtheorem{cor}[thm]{Corollary}
\newtheorem{conj}[thm]{Conjecture}
\newtheorem{thmi}{Theorem}     % separate numbering for the introduction
\newtheorem{conji}{Conjecture} % separate numbering for the introduction
\newtheorem{claimi}{Claim} % separate numbering

\theoremstyle{definition} 
\newtheorem{defin}[thm]{Definition}
\newtheorem{cons}[thm]{Construction}
\newtheorem{remark}[thm]{Remark}
\newtheorem{variant}[thm]{Variant}
\newtheorem{example}[thm]{Example}

% script letters
\newcommand{\cA}{\mathcal{A}}
\newcommand{\cB}{\mathscr B}
\newcommand{\cC}{\mathcal{C}}
\newcommand{\cD}{\mathscr D}
\newcommand{\cE}{\mathscr E}
\newcommand{\cF}{\mathscr F}
\newcommand{\cG}{\mathscr G}
\newcommand{\cH}{\mathscr H}
\newcommand{\cI}{\mathscr I}
\newcommand{\cJ}{\mathscr J}
\newcommand{\cK}{\mathscr K}
\newcommand{\cL}{\mathscr L}
\newcommand{\cM}{\mathscr M}
\newcommand{\cN}{\mathscr N}
\newcommand{\cO}{\mathscr O}
\newcommand{\cP}{\mathscr P}
\newcommand{\cR}{\mathcal{R}}
\newcommand{\cS}{\mathscr S}
\newcommand{\cQ}{\mathscr Q}
\newcommand{\cT}{\mathscr T}
\newcommand{\cU}{\mathscr U}
\newcommand{\cW}{\mathscr W}
\newcommand{\cX}{\mathscr X}
\newcommand{\cY}{\mathscr Y}
\newcommand{\cZ}{\mathscr Z}

% boldface letters
\newcommand{\bb}[1]{\mathbf{#1}} 
\newcommand{\FF}{\bb{F}}
\newcommand{\GG}{\bb{G}}
\newcommand{\NN}{\bb{N}}
\newcommand{\PP}{\bb{P}}
\newcommand{\QQ}{\bb{Q}}
\newcommand{\RR}{\bb{R}}
\newcommand{\ZZ}{\bb{Z}}
\newcommand{\bt}{\mathbf}

%raccourcis
\newcommand{\depth}{\mathrm{depth}}
\newcommand{\height}{\mathrm{ht}}
\newcommand{\Id}{\mathrm{Id}}
\newcommand{\Mcomp}{M_{\bullet}}
\newcommand{\Ncomp}{N_{\bullet}}
\newcommand{\A}{\mathbb{A}}
\newcommand{\C}{\mathbb{C}}
\newcommand{\Z}{\mathbb{Z}}
\newcommand{\Q}{\mathbb{Q}}
\newcommand{\N}{\mathbb{N}}
\newcommand{\K}{\mathbb{K}}
\newcommand{\cechC}{\mathrm{\textit{\v{C}}}}
\newcommand{\cechH}{\mathrm{\textit{\v{H}}}}
\newcommand{\injdim}{\mathrm{injdim}}
\newcommand{\pd}{\mathrm{pd}}
\newcommand{\aideal}{\mathfrak{a}}
\newcommand{\mideal}{\mathfrak{m}}
\newcommand{\nideal}{\mathfrak{n}}
\newcommand{\pideal}{\mathfrak{p}}
\newcommand{\qideal}{\mathfrak{q}}
\newcommand{\rank}{\mathrm{rank}}


% symbols
\renewcommand{\phi}{\varphi}
\newcommand{\isom}{\simeq} 
\newcommand{\piet}{\pi^{\acute{e}t}_{1}}
\newcommand{\et}{{\rm\acute{e}t}}
\newcommand{\cat}[1]{{\normalfont\textbf{#1}}}  % categories
\newcommand{\ra}{\longrightarrow}    % to be used instead of \to in displayed formulas
\newcommand{\isomto}{\xrightarrow{\,\smash{\raisebox{-0.65ex}{\ensuremath{\scriptstyle\sim}}}\,}}
\renewcommand{\bigwedge}{\mbox{\large $\wedge$}}


% operators
\DeclareMathOperator{\Alb}{Alb}
\DeclareMathOperator{\Art}{\cat{Art}} 
\DeclareMathOperator{\Aut}{Aut}
\DeclareMathOperator{\Bl}{Bl}
\DeclareMathOperator{\Def}{Def}
\DeclareMathOperator{\Ext}{Ext}
\DeclareMathOperator{\Gr}{Gr}

%basic stuff
\newcommand{\tens}[1]{%
  \mathbin{\mathop{\otimes}\displaylimits_{#1}}%
}
\DeclareMathOperator{\Hom}{Hom}
\DeclareMathOperator{\Pic}{Pic}
\DeclareMathOperator{\Proj}{Proj}
\DeclareMathOperator{\Sing}{Sing}
\DeclareMathOperator{\Spec}{Spec}
\DeclareMathOperator{\Sym}{Sym}
\DeclareMathOperator{\Tor}{Tor}
\DeclareMathOperator{\Ass}{Ass}
\DeclareMathOperator{\Ann}{Ann}
\DeclareMathOperator{\Nil}{Nil}
\DeclareMathOperator{\Supp}{Supp}
\DeclareMathOperator{\Frac}{Frac}
\DeclareMathOperator{\Ob}{Ob}
\DeclareMathOperator{\Mor}{Mor}
\DeclareMathOperator{\Fun}{Fun}
\DeclareMathOperator{\Nat}{Nat}
% sheaf Hom etc
\newcommand{\cExt}{{\mathscr E}\kern -.5pt xt}
\newcommand{\cHom}{\mathscr{H}\kern -.5pt om}
\newcommand{\cEnd}{{\mathscr E}nd}

% other
\newcommand{\stacksproj}[1]{{\cite[Tag~\href{http://stacks.math.columbia.edu/tag/#1}{#1}]{stacks-project}}}

% tildes
% - for uppercase letters X Y Z F D U R A ... use \wt X etc
% - for lowercase letters \pi ... use \swt \pi 
% - with the exception of "f" and "s" (used in one proof), where I used \tilde
% - for script I, R (ideal used in section 3.4) use \wtcI, \wtcR
% - for complex expressions e.g. X\times Y use \widetilde{...}

\newcommand{\wt}[1]{{\mathchoice%
  {\raisebox{1.5ex}{\resizebox{1.7ex}{!}{{}\hphantom{i}\ensuremath{{\sim}}}} \hspace{-1.7ex}{#1}}%
  {\smash{\raisebox{1.5ex}{\resizebox{1.7ex}{!}{{}\hphantom{i}\ensuremath{{\sim}}}}\hspace{-1.7ex}{#1 }}\vphantom{\tilde I}}%
  {\raisebox{1.1ex}{\resizebox{1.3ex}{!}{{}\hphantom{i}\ensuremath{{\sim}}}}\hspace{-1.3ex}{#1}}%
  {\raisebox{0.8ex}{\resizebox{1ex}{!}{{}\hphantom{i}\ensuremath{{\sim}}}}\hspace{-1ex}{#1}}%
}}

\newcommand{\swt}[1]{{\mathchoice%
  {\raisebox{0.9ex}{\resizebox{1.2ex}{!}{\ensuremath{{\sim}}}}\hspace{-1.4ex}{#1}}%
  {\smash{\raisebox{0.9ex}{\resizebox{1.2ex}{!}{\ensuremath{{\sim}}}}\hspace{-1.4ex}{#1 }}\vphantom{I}}%
  {\raisebox{0.7ex}{\resizebox{0.8ex}{!}{\ensuremath{{\sim}}}}\hspace{-0.9ex}{#1}}%
  {\raisebox{0.5ex}{\resizebox{1ex}{!}{{}\hphantom{i}\ensuremath{{\sim}}}}\hspace{-1ex}{#1}}%
}}

\newcommand{\wtcI}{\smash{\raisebox{1.5ex}{\hspace{0.7ex}\resizebox{1.2ex}{!}{\ensuremath{{\sim}}}}\hspace{-2.1ex}{\cI}}\vphantom{I}}

\newcommand{\wtcR}{\smash{\raisebox{1.5ex}{\hspace{0.7ex}\resizebox{1.2ex}{!}{\ensuremath{{\sim}}}}\hspace{-2.1ex}{\cR}}\vphantom{I}}

%%%%%%%%%%%%%%%%%%%%%%%%%%%%%%%%%%%%%%%%%%%


\begin{document}
\title{Homotopical Algebra\\ Hand in 2}
\author{Christophe Marciot}
\maketitle

\section*{Notation:}
From now on in this document, we will use the followind notation. For $\xymatrix{c'\ar[r]^{f} & c}\in\Ob\cC/c$, we will note $(c',f)$ and $(c'',h,c')\in\cC((c'',g),(c',f))$ for every morphisms from $(c'',g)$ to $(c',f)$ 
\begin{displaymath}
	\xymatrix{
		c'' \ar[rd]_g \ar[rr]^h & & c' \ar[ld]^f\\
		& c
	}
\end{displaymath}
This may be a confusing notation since it does not contain the information of $f$ or $g$, but it will be made clear from context.

\section*{Exercise 7:}
\subsection*{Part (a):}

Let $d\in\Ob\cD$ and let us define the functor $\bar{G}:\cD/d\ra\cC/G(d)$. We set $\bar{G}(d',f)=(G(d'),G(f))$ on objects of $\cD/d$ and $\bar{G}((d'',h,d'))=(G(d''),G(h),G(d'))$ for the a morphism from $(d'',g)$ to$(d',f)$. Observe that $(G(d''),G(h),G(d'))$ is indeed a morphism since we have that the commutativity of the diagram 
\begin{displaymath}
	\xymatrix{
		d'' \ar[rd]_g \ar[rr]^h & & d' \ar[ld]^f\\
		& d
	}
\end{displaymath}
implies the commutativity of the diagram
\begin{displaymath}
	\xymatrix{
		G(d'') \ar[rd]_{G(g)} \ar[rr]^{G(h)} & & {G(d')} \ar[ld]^{G(f)}\\
		& G(d)
	}
\end{displaymath}
  Now we need to prove that this defines a functor. Let us compute the image of $\Id_{(d',f)}$ for $(d',f)\in\Ob\cD/d$. We have 
\begin{displaymath}
	\bar{G}(\Id_{(d',f)})=(G(d'),G(\Id_{d'}),G(d'))=(G(d'),\Id_{G(d')},G(d'))=\Id_{\bar{G}(d',f)}.
\end{displaymath}
	Now let us compute the image of a composition. For $(d''',i,d'')$ and $(d'',j,d')$, we have 
\begin{align*}
	\bar{G}((d'',j,d')(d''',i,d''))  =&   \bar{G}(d''',ji,d')\\
							=& (G(d'''),G(ji),G(d'))\\
	                                      =& (G(d'''),G(j)G(i),G(d'))\\
	                                      =& (G(d''),G(j),G(d'))(G(d'''),G(i),G(d''))\\
	                                      =& \bar{G}(d'',j,d')\bar{G}(d''',i,d'').
\end{align*}
This allows to say that $\bar{G}$ is a functor.

\subsection*{Part (b):}

As we know that the pair $(F,G)$ is a pair of adjoint functors, we know there exists a natural isomorphism
\begin{displaymath}
	\tau:\cC(-,G(-))\Longrightarrow\cD(F(-),-).
\end{displaymath}
We set the following functor $\bar{F}:\cC/G(d)\ra\cD/d$ the following way. On the objects, we set $\bar{F}(c,f)=(F(c),\tau_{(c,d)}(f))$ and on morphisms $\bar{F}(c',h,c)=(F(c'),F(h),F(c))$ for a morphism form $(c,f)$ to $(c',g)$. We need now to prove that $(F(c'),F(h),F(c))$ is indeed a morphism from $(F(c'),\tau_{(c',d)}(g))$ to $(F(c),\tau_{(c,d)}(f))$, i.e. we need to prove that the triangle
\begin{displaymath}
	\xymatrix{
		F(c') \ar[rd]_{\tau_{(c',d)}(g)} \ar[rr]^{F(h)} & & F(c) \ar[ld]^{\tau_{(c,d)}(f)}\\
		& d
	} 
\end{displaymath}
Note first that since $(c',h,c)$ is a morphism from $(c,f)$ to $(c',g)$, we have $fh=g$. The fact that $\tau$ is a natural transformation gives us that 
\begin{align*}
	\cD(F(-),-)(h^{op},\Id_d)\tau_{(c,d)}(f)&=\Id_d\tau_{(c,d)}(f)F(h)=\tau_{(c,d)}(f)F(h)  \\
	\tau_{(c',d)}\cC(-,G(-))(h^{op},\Id_d)(f)&=\tau_{(c',d)}(G(\Id_d)fh)=\tau_{(c',d)}(fh)=\tau_{(c'd)}(g)
\end{align*}
	those two ligns are equal and so we conclude that the last triangle commutes.\\
Now we need to prove that $\bar{F}$ is a functor. First consider $\Id_{(c,f)}$ and let us compute its image. We have 
\begin{align*}
	\bar{F}(\Id_{(c,f)})=(F(c),F(\Id_{F(c)},F(c)))=\Id_{F(c)}=\Id_{\bar{F}(c,f)}.
\end{align*}
Let us compute the image of a composition. Suppose, we have $(c'',i,c')$ and $(c',j,c)$. Then 
\begin{align*}
	\bar{F}((c',j,c)(c'',i,c'))&=\bar{F}(c'',ji,c)\\
						&= (F(c''),F(ji),F(c))\\
						&= (F(c''),F(j)F(i),F(c))\\
						&= (F(c''),F(j),F(c'))(F(c'))(F(c'),F(i),F(c))\\
						&= \bar{F}(c'',j,c')\bar{F}(c',i,c)
\end{align*}
	and thus we conclude that $\bar{F}$ is a functor. What is left is to prove that the pair $(\bar{F},\bar{G})$ is a pair of adjoint functors. Let $(c,f)\in\Ob\cC/G(d)$ and $(d',g)\in\Ob\cD$. We would like to find a map from $\cC/G(d)((c,f),(G(d'),G(g)))$ to $\cD/d((F(c),\tau_{(c,d)}(f)),(d',g))$, that is if we have the commutative triangle 
\begin{displaymath}
	\xymatrix{
		c \ar[rd]_f \ar[rr]^h & & G(d') \ar[ld]^{G(g)}\\
		& G(d)
	} 
\end{displaymath}
we would like to send $h$ to a map $\bar{h}$ such that the triangle 
\begin{displaymath}
	\xymatrix{
		F(c) \ar[rd]_{\tau_{(c,d)}(f)} \ar[rr]^{\bar{h}} & & d' \ar[ld]^{g}\\
		& d
	} 
\end{displaymath}
commutes. Recall that since $\tau$ is a natural isomorphism, the square
\begin{displaymath}
	\xymatrix{
		\cC(G(d'),G(d')) \ar[rr]^{\tau_{(G(d'),d')}} \ar[d]_{\bar{\cC}(h^{op},g)} & & \cD(FG(d'),d') \ar[d]^{\bar{\cD}(h^{op},g)}\\
		\cC(c',G(d)) \ar[rr]_{\tau_{(c,d)}} & & \cD(F(c'),d)
	}
\end{displaymath}
commutes, where $\bar{\cC}=\cC(-,G(-))$ and $\bar{\cD}=\cD(F(-),-)$. Note that we then have that 
\begin{align*}
	\tau_{(c,d)}(f) =& \tau_{(c,d)}(G(g)h)= \tau_{(c,d)}\bar{\cC}(h^{op},g)(\Id_{G(d')})\\
				   =& \bar{\cD}(h^{op},g)\tau_{(G(d'),d')}(\Id_{G(d')})\\
				   =& g\tau_{(G(d'),d')}(\Id_{G(d')})F(h)
\end{align*}
and so our guess is $\bar{h}=\tau'F(h)$, where $\tau'=\tau_{(G(d'),(d'))}$. Let us denote the family of maps obtained by $\theta_{((c,f),(d',g)}$ and let us denote by $\theta$ the natural transformation given by this family. %We called $\theta$ a natural transformation, but we actually still have to prove that it is one by showing that the square 
\end{document}
