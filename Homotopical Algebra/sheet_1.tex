%%%%%%%%%%%%%%%%%%%%%%%%%%%%%%%%%%%%%%%%%%%

\documentclass[11pt, a4paper, twoside]{article}
\usepackage[utf8]{inputenc}
\usepackage[T1]{fontenc}
\usepackage[francais]{babel}
\usepackage{lmodern}
\usepackage[margin=0.8in]{geometry}

\usepackage{amsmath}
\usepackage{amssymb}
\usepackage{amsthm}
\usepackage{mathrsfs}
\usepackage{mathtools}
\usepackage{enumitem}
\usepackage{graphicx}
\usepackage[all]{xy}

\DeclarePairedDelimiter\ceil{\lceil}{\rceil}
\DeclarePairedDelimiter\floor{\lfloor}{\rfloor}

% environments
\newtheorem{thm}{Theorem}[subsection]
\newtheorem{lemma}[thm]{Lemma}
\newtheorem{prop}[thm]{Proposition}
\newtheorem{cor}[thm]{Corollary}
\newtheorem{conj}[thm]{Conjecture}
\newtheorem{thmi}{Theorem}     % separate numbering for the introduction
\newtheorem{conji}{Conjecture} % separate numbering for the introduction
\newtheorem{claimi}{Claim} % separate numbering

\theoremstyle{definition} 
\newtheorem{defin}[thm]{Definition}
\newtheorem{cons}[thm]{Construction}
\newtheorem{remark}[thm]{Remark}
\newtheorem{variant}[thm]{Variant}
\newtheorem{example}[thm]{Example}

% script letters
\newcommand{\cA}{\mathcal{A}}
\newcommand{\cB}{\mathscr B}
\newcommand{\cC}{\mathcal{C}}
\newcommand{\cD}{\mathscr D}
\newcommand{\cE}{\mathscr E}
\newcommand{\cF}{\mathscr F}
\newcommand{\cG}{\mathscr G}
\newcommand{\cH}{\mathscr H}
\newcommand{\cI}{\mathscr I}
\newcommand{\cJ}{\mathscr J}
\newcommand{\cK}{\mathscr K}
\newcommand{\cL}{\mathscr L}
\newcommand{\cM}{\mathscr M}
\newcommand{\cN}{\mathscr N}
\newcommand{\cO}{\mathscr O}
\newcommand{\cP}{\mathscr P}
\newcommand{\cR}{\mathcal{R}}
\newcommand{\cS}{\mathscr S}
\newcommand{\cQ}{\mathscr Q}
\newcommand{\cT}{\mathscr T}
\newcommand{\cU}{\mathscr U}
\newcommand{\cW}{\mathscr W}
\newcommand{\cX}{\mathscr X}
\newcommand{\cY}{\mathscr Y}
\newcommand{\cZ}{\mathscr Z}

% boldface letters
\newcommand{\bb}[1]{\mathbf{#1}} 
\newcommand{\FF}{\bb{F}}
\newcommand{\GG}{\bb{G}}
\newcommand{\NN}{\bb{N}}
\newcommand{\PP}{\bb{P}}
\newcommand{\QQ}{\bb{Q}}
\newcommand{\RR}{\bb{R}}
\newcommand{\ZZ}{\bb{Z}}
\newcommand{\bt}{\mathbf}

%raccourcis
\newcommand{\depth}{\mathrm{depth}}
\newcommand{\height}{\mathrm{ht}}
\newcommand{\Id}{\mathrm{Id}}
\newcommand{\Mcomp}{M_{\bullet}}
\newcommand{\Ncomp}{N_{\bullet}}
\newcommand{\A}{\mathbb{A}}
\newcommand{\C}{\mathbb{C}}
\newcommand{\Z}{\mathbb{Z}}
\newcommand{\Q}{\mathbb{Q}}
\newcommand{\N}{\mathbb{N}}
\newcommand{\K}{\mathbb{K}}
\newcommand{\cechC}{\mathrm{\textit{\v{C}}}}
\newcommand{\cechH}{\mathrm{\textit{\v{H}}}}
\newcommand{\injdim}{\mathrm{injdim}}
\newcommand{\pd}{\mathrm{pd}}
\newcommand{\aideal}{\mathfrak{a}}
\newcommand{\mideal}{\mathfrak{m}}
\newcommand{\nideal}{\mathfrak{n}}
\newcommand{\pideal}{\mathfrak{p}}
\newcommand{\qideal}{\mathfrak{q}}
\newcommand{\rank}{\mathrm{rank}}


% symbols
\renewcommand{\phi}{\varphi}
\newcommand{\isom}{\simeq} 
\newcommand{\piet}{\pi^{\acute{e}t}_{1}}
\newcommand{\et}{{\rm\acute{e}t}}
\newcommand{\cat}[1]{{\normalfont\textbf{#1}}}  % categories
\newcommand{\ra}{\longrightarrow}    % to be used instead of \to in displayed formulas
\newcommand{\isomto}{\xrightarrow{\,\smash{\raisebox{-0.65ex}{\ensuremath{\scriptstyle\sim}}}\,}}
\renewcommand{\bigwedge}{\mbox{\large $\wedge$}}


% operators
\DeclareMathOperator{\Alb}{Alb}
\DeclareMathOperator{\Art}{\cat{Art}} 
\DeclareMathOperator{\Aut}{Aut}
\DeclareMathOperator{\Bl}{Bl}
\DeclareMathOperator{\Def}{Def}
\DeclareMathOperator{\Ext}{Ext}
\DeclareMathOperator{\Gr}{Gr}

%basic stuff
\newcommand{\tens}[1]{%
  \mathbin{\mathop{\otimes}\displaylimits_{#1}}%
}
\DeclareMathOperator{\Hom}{Hom}
\DeclareMathOperator{\Pic}{Pic}
\DeclareMathOperator{\Proj}{Proj}
\DeclareMathOperator{\Sing}{Sing}
\DeclareMathOperator{\Spec}{Spec}
\DeclareMathOperator{\Sym}{Sym}
\DeclareMathOperator{\Tor}{Tor}
\DeclareMathOperator{\Ass}{Ass}
\DeclareMathOperator{\Ann}{Ann}
\DeclareMathOperator{\Nil}{Nil}
\DeclareMathOperator{\Supp}{Supp}
\DeclareMathOperator{\Frac}{Frac}
\DeclareMathOperator{\Ob}{Ob}
% sheaf Hom etc
\newcommand{\cExt}{{\mathscr E}\kern -.5pt xt}
\newcommand{\cHom}{\mathscr{H}\kern -.5pt om}
\newcommand{\cEnd}{{\mathscr E}nd}

% other
\newcommand{\stacksproj}[1]{{\cite[Tag~\href{http://stacks.math.columbia.edu/tag/#1}{#1}]{stacks-project}}}

% tildes
% - for uppercase letters X Y Z F D U R A ... use \wt X etc
% - for lowercase letters \pi ... use \swt \pi 
% - with the exception of "f" and "s" (used in one proof), where I used \tilde
% - for script I, R (ideal used in section 3.4) use \wtcI, \wtcR
% - for complex expressions e.g. X\times Y use \widetilde{...}

\newcommand{\wt}[1]{{\mathchoice%
  {\raisebox{1.5ex}{\resizebox{1.7ex}{!}{{}\hphantom{i}\ensuremath{{\sim}}}} \hspace{-1.7ex}{#1}}%
  {\smash{\raisebox{1.5ex}{\resizebox{1.7ex}{!}{{}\hphantom{i}\ensuremath{{\sim}}}}\hspace{-1.7ex}{#1 }}\vphantom{\tilde I}}%
  {\raisebox{1.1ex}{\resizebox{1.3ex}{!}{{}\hphantom{i}\ensuremath{{\sim}}}}\hspace{-1.3ex}{#1}}%
  {\raisebox{0.8ex}{\resizebox{1ex}{!}{{}\hphantom{i}\ensuremath{{\sim}}}}\hspace{-1ex}{#1}}%
}}

\newcommand{\swt}[1]{{\mathchoice%
  {\raisebox{0.9ex}{\resizebox{1.2ex}{!}{\ensuremath{{\sim}}}}\hspace{-1.4ex}{#1}}%
  {\smash{\raisebox{0.9ex}{\resizebox{1.2ex}{!}{\ensuremath{{\sim}}}}\hspace{-1.4ex}{#1 }}\vphantom{I}}%
  {\raisebox{0.7ex}{\resizebox{0.8ex}{!}{\ensuremath{{\sim}}}}\hspace{-0.9ex}{#1}}%
  {\raisebox{0.5ex}{\resizebox{1ex}{!}{{}\hphantom{i}\ensuremath{{\sim}}}}\hspace{-1ex}{#1}}%
}}

\newcommand{\wtcI}{\smash{\raisebox{1.5ex}{\hspace{0.7ex}\resizebox{1.2ex}{!}{\ensuremath{{\sim}}}}\hspace{-2.1ex}{\cI}}\vphantom{I}}

\newcommand{\wtcR}{\smash{\raisebox{1.5ex}{\hspace{0.7ex}\resizebox{1.2ex}{!}{\ensuremath{{\sim}}}}\hspace{-2.1ex}{\cR}}\vphantom{I}}

%%%%%%%%%%%%%%%%%%%%%%%%%%%%%%%%%%%%%%%%%%%


\begin{document}
\title{Hand in 1}
\author{Christophe Marciot}
\maketitle

\section*{Exercise 6}
Suppose we have a natural transformation $\tau:F\rightarrow G$. We want to associate to $\tau$ a functor $H_\tau:\cC\times[1]\rightarrow\cD$ in a meaningful way. In fact we want that this association to remind us of the notion of homotopy in regular homotopy theory. \\
We propose the follwing definition. For an object $(c,x)\in \Ob\cC\times[1]$ and an arrow $(f,p)\in \Ob\cC\times[1]((c,x),(c',x'))$ we set their image by $H_\tau$:
	\begin{align*}
		H_\tau(c,x)=
			\begin{cases}
				F(c) &  \mathrm{if}\ x=0,\\
				G(c) & \mathrm{if}\ x=1
			\end{cases}
	\end{align*}
and 
	\begin{align*}
		H_\tau(f,p)=
			\begin{cases}
				F(f)  & \mathrm{if}\ p=0\leq0,\\
				\tau_{c'}F(f)=G(f)\tau_c & \mathrm{if}\ p=0\leq1,\\
				G(f) & \mathrm{if}\ p=1\leq1.
			\end{cases}
	\end{align*}
Note that the equalitiy in the second line is due to the fact that $\tau$ is a natural transformation.\\
 First, we need to prove that $H_\tau$ is a functor. Let us consider the identity $\Id_{(c,x)}=(\Id_c,x\leq x)$. We want to prove that $H_\tau(\Id_{(c,x)})=\Id_{H_\tau(c,x)}$. We distinguish two cases:
 	\begin{itemize}
		\item[$\bullet$] \underline{If $x=0$}: Then we get that $H_\tau(\Id_{(c,0)})=H_\tau(Id_c,0\leq0)=F(\Id_c)=\Id_{F(c)}=\Id_{H_\tau(c,0)}.$
		\item[$\bullet$] \underline{If $x=1$}: Then we get that $H_\tau(\Id_{(c,1)})=H_\tau(Id_c,1\leq1)=G(\Id_c)=\Id_{G(c)}=\Id_{H_\tau(c,1)}.$
	\end{itemize}
 Note that the third equalities are given by the fact that $F$ and $G$ are functors. Secondly, we need to prove that $H_\tau$ respects composition. Suppose we have $\xymatrix{(c,x)\ar[r]^{(f,p)}& (c',x')\ar[r]^{(f',p')}& (c'',x'')}$. We consider the following cases:
 	\begin{itemize}
		\item[$\bullet$] \underline{If $p=p'=0\leq0$}: We have $$H_\tau((f',0\leq0)(f,0\leq0))=H_\tau(f'f,0\leq0)=F(f'f)=F(f')F(f)=H_\tau(f',0\leq0)H_\tau(f,0\leq0).$$
		\item[$\bullet$] \underline{If $p=p'=1\leq0$}: We have $$H_\tau((f',1\leq1)(f,1\leq1))=H_\tau(f'f,1\leq1)=F(f'f)=F(f')F(f)=H_\tau(f',1\leq1)H_\tau(f,1\leq1).$$
		\item[$\bullet$] \underline{If $p=0\leq0$ and $p'=0\leq1$}: We have $$H_\tau((f',0\leq1)(f,0\leq0))=H_\tau(f'f,0\leq1)=\tau_{c''}F(f'f)=(\tau_{c''}F(f'))F(f)=H_\tau(f',0\leq1)H_\tau(f,0\leq0).$$
		\item[$\bullet$] \underline{If $p=0\leq1$ and $p'=1\leq1$}:  We have $$H_\tau((f',1\leq1)(f,0\leq1))=H_\tau(f'f,0\leq1)=G(f'f)\tau_{c}=G(f')(G(f)\tau_{c})=H_\tau(f',1\leq1)H_\tau(f,0\leq1).$$
	\end{itemize}
	(Note that the third and fourth cases are actually the same). This proves that $H_\tau$ is a functor. Now, thirdly, we need to prove that $H_\tau$ respects the wanted commutative property. We have that
	$$\begin{cases}
		 H_\tau\circ\Id_\cC\times\{0\}(c)=H_\tau(c,0)=F(c), \\
		 H_\tau\circ\Id_\cC\times\{1\}(c)=H_\tau(c,1)=G(c), \\
		 H_\tau\circ\Id_\cC\times\{0\}(f)=H_\tau(f,0\leq0)=F(f), \\
		 H_\tau\circ\Id_\cC\times\{1\}(f)=H_\tau(f,1\leq1)=G(f),
	\end{cases}$$
	for each $c\in\Ob\cC$ and $f\in\Mor\cC$. So the commutativity criterion is respected.\newpage
	
Now, let $H\in\Fun_{F,G}(\cC\times[1],\cD)$. We want to associate to $H$ a natural transformation $\tau_H$ from $F$ to $G$. The natural thing to do seems to set $(\tau_H)_c=H(\Id_c,0\leq1)$ for each $c\in\Ob\cC$. Note that the commutative properties of elements of $\Fun_{F,G}(\cC\times[1],\cD$) gives that the domain and the codomain of $(\tau_H)_c$ are 
	\begin{align*}
	\dom(\tau_H)_c =& \dom H(\Id_c,0\leq1) = \dom \{H(\Id_c,0\leq1)H(\Id_c,0\leq0)\} \\
				    =& \dom H(\Id_c,0\leq0) = \dom F(\Id_c) = \dom\Id_{F(c)} = F(c),\\
	\cod(\tau_H)_c =& \cod H(\Id_c,0\leq1) = \cod \{H(\Id_c,1\leq1)H(\Id_c,0\leq1)\}\\
				   =& \cod H(\Id_c,1\leq1) = \cod G(\Id_c) = \cod\Id_{G(c)} = G(c)
	\end{align*}
	and so $(\tau_H)_c$ is indeed a morphism from $F(c)$ to $G(c)$. We then need to show that this defines a natural transformation. Let $c,d\in\Ob\cC$ and $f\in\cC(c,d)$. We need to show that 
	\begin{displaymath}
		\xymatrix{
			F(c) \ar[r]^{(\tau_H)_c} \ar[d]_{F(f)} & G(c) \ar[d]^{G(f)}\\
			F(d) \ar[r]_{(\tau_H)_d} & G(d)
		}
	\end{displaymath}
commutes. This said we have that 
	\begin{displaymath}
		G(f)(\tau_H)_c=H(f,1\leq1)H(\Id_c,0\leq1)=H(f,0\leq1)=H(\Id_d,0\leq1)H(f,0\leq0)=(\tau_H)_dF(f)
	\end{displaymath} 
and thus the diagram commutes and we have that $\tau_H$ is a natural transformation.\\\\
Lastly, we need to prove that these associations are inverses of each other, i.e. $H_{\tau_H}=H$ for each $H\in\Fun_{F,G}(\cC\times[1],\cD)$ and $\tau_{H_\tau}=\tau$ for each $\tau\in\Nat(F,G)$. For $H\in\Fun_{F,G}(\cC\times[1],\cD)$, $(c,x),(d,y)\in\Ob\cC\times[1]$ and $(f,p)\in\Ob\cC\times[1]((c,x),(d,y))$, we have that 
	\begin{align*}
		H_{\tau_H}(c,x) &=
			\begin{cases}
				F(c)=H(c,0) & \mathrm{if}\ x=0,\\
				G(c)=H(c,1) & \mathrm{if}\ x=1.
			\end{cases}\\
		H_{\tau_H}(f,p) &=
			\begin{cases}
				F(f) = H(f,0\leq0) & \mathrm{if}\ p=0\leq0,\\
				(\tau_H)_dF(f)=H(\Id_d,0\leq1)H(f,0\leq0)=H(f,0\leq1) & \mathrm{if}\ p=0\leq1,\\
				G(f)  = H(f,1\leq1) & \mathrm{if}\ p=1\leq1.
			\end{cases}
		\end{align*}
	We clearly see that $H_{\tau_H}$ is equal to $H$. Lastly, let $\tau\in\Nat(F,G)$. Then 
		\begin{displaymath}
			(\tau_{H_\tau})_c=H_\tau(\Id_c,0\leq1)=\tau_c
		\end{displaymath}
			for each $c\in\Ob\cC$ and thus we get that $\tau_{H_\tau}$ is equal to $\tau$. We can then conclude that those assignations are inverses to each other and define bijective maps between $\Nat(F,G)$ and $\Fun_{F,G}(\cC\times[1],\cD)$.
	
\end{document}
