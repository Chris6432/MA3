% environments
\newtheorem{thm}{Theorem}[subsection]
\newtheorem{lemma}[thm]{Lemma}
\newtheorem{prop}[thm]{Proposition}
\newtheorem{cor}[thm]{Corollary}
\newtheorem{conj}[thm]{Conjecture}
\newtheorem{thmi}{Theorem}     % separate numbering for the introduction
\newtheorem{conji}{Conjecture} % separate numbering for the introduction
\newtheorem{claimi}{Claim} % separate numbering

\theoremstyle{definition} 
\newtheorem{defin}[thm]{Definition}
\newtheorem{cons}[thm]{Construction}
\newtheorem{remark}[thm]{Remark}
\newtheorem{variant}[thm]{Variant}
\newtheorem{example}[thm]{Example}

% script letters
\newcommand{\cA}{\mathcal{A}}
\newcommand{\cB}{\mathscr B}
\newcommand{\cC}{\mathcal{C}}
\newcommand{\cD}{\mathscr D}
\newcommand{\cE}{\mathscr E}
\newcommand{\cF}{\mathscr F}
\newcommand{\cG}{\mathscr G}
\newcommand{\cH}{\mathscr H}
\newcommand{\cI}{\mathscr I}
\newcommand{\cJ}{\mathscr J}
\newcommand{\cK}{\mathscr K}
\newcommand{\cL}{\mathscr L}
\newcommand{\cM}{\mathscr M}
\newcommand{\cN}{\mathscr N}
\newcommand{\cO}{\mathscr O}
\newcommand{\cP}{\mathscr P}
\newcommand{\cR}{\mathcal{R}}
\newcommand{\cS}{\mathscr S}
\newcommand{\cQ}{\mathscr Q}
\newcommand{\cT}{\mathscr T}
\newcommand{\cU}{\mathscr U}
\newcommand{\cW}{\mathscr W}
\newcommand{\cX}{\mathscr X}
\newcommand{\cY}{\mathscr Y}
\newcommand{\cZ}{\mathscr Z}

% boldface letters
\newcommand{\bb}[1]{\mathbf{#1}} 
\newcommand{\FF}{\bb{F}}
\newcommand{\GG}{\bb{G}}
\newcommand{\NN}{\bb{N}}
\newcommand{\PP}{\bb{P}}
\newcommand{\QQ}{\bb{Q}}
\newcommand{\RR}{\bb{R}}
\newcommand{\ZZ}{\bb{Z}}
\newcommand{\bt}{\mathbf}

%raccourcis
\newcommand{\depth}{\mathrm{depth}}
\newcommand{\height}{\mathrm{ht}}
\newcommand{\Id}{\mathrm{Id}}
\newcommand{\Mcomp}{M_{\bullet}}
\newcommand{\Ncomp}{N_{\bullet}}
\newcommand{\A}{\mathbb{A}}
\newcommand{\C}{\mathbb{C}}
\newcommand{\Z}{\mathbb{Z}}
\newcommand{\Q}{\mathbb{Q}}
\newcommand{\N}{\mathbb{N}}
\newcommand{\K}{\mathbb{K}}
\newcommand{\cechC}{\mathrm{\textit{\v{C}}}}
\newcommand{\cechH}{\mathrm{\textit{\v{H}}}}
\newcommand{\injdim}{\mathrm{injdim}}
\newcommand{\pd}{\mathrm{pd}}
\newcommand{\aideal}{\mathfrak{a}}
\newcommand{\mideal}{\mathfrak{m}}
\newcommand{\nideal}{\mathfrak{n}}
\newcommand{\pideal}{\mathfrak{p}}
\newcommand{\qideal}{\mathfrak{q}}
\newcommand{\rank}{\mathrm{rank}}


% symbols
\renewcommand{\phi}{\varphi}
\newcommand{\isom}{\simeq} 
\newcommand{\piet}{\pi^{\acute{e}t}_{1}}
\newcommand{\et}{{\rm\acute{e}t}}
\newcommand{\cat}[1]{{\normalfont\textbf{#1}}}  % categories
\newcommand{\ra}{\longrightarrow}    % to be used instead of \to in displayed formulas
\newcommand{\isomto}{\xrightarrow{\,\smash{\raisebox{-0.65ex}{\ensuremath{\scriptstyle\sim}}}\,}}
\renewcommand{\bigwedge}{\mbox{\large $\wedge$}}


% operators
\DeclareMathOperator{\Alb}{Alb}
\DeclareMathOperator{\Art}{\cat{Art}} 
\DeclareMathOperator{\Aut}{Aut}
\DeclareMathOperator{\Bl}{Bl}
\DeclareMathOperator{\Def}{Def}
\DeclareMathOperator{\Ext}{Ext}
\DeclareMathOperator{\Gr}{Gr}

%basic stuff
\newcommand{\tens}[1]{%
  \mathbin{\mathop{\otimes}\displaylimits_{#1}}%
}
\DeclareMathOperator{\Hom}{Hom}
\DeclareMathOperator{\Pic}{Pic}
\DeclareMathOperator{\Proj}{Proj}
\DeclareMathOperator{\Sing}{Sing}
\DeclareMathOperator{\Spec}{Spec}
\DeclareMathOperator{\Sym}{Sym}
\DeclareMathOperator{\Tor}{Tor}
\DeclareMathOperator{\Ass}{Ass}
\DeclareMathOperator{\Ann}{Ann}
\DeclareMathOperator{\Nil}{Nil}
\DeclareMathOperator{\Supp}{Supp}
\DeclareMathOperator{\Frac}{Frac}
\DeclareMathOperator{\Ob}{Ob}
\DeclareMathOperator{\Mor}{Mor}
\DeclareMathOperator{\Fun}{Fun}
\DeclareMathOperator{\Nat}{Nat}
% sheaf Hom etc
\newcommand{\cExt}{{\mathscr E}\kern -.5pt xt}
\newcommand{\cHom}{\mathscr{H}\kern -.5pt om}
\newcommand{\cEnd}{{\mathscr E}nd}

% other
\newcommand{\stacksproj}[1]{{\cite[Tag~\href{http://stacks.math.columbia.edu/tag/#1}{#1}]{stacks-project}}}

% tildes
% - for uppercase letters X Y Z F D U R A ... use \wt X etc
% - for lowercase letters \pi ... use \swt \pi 
% - with the exception of "f" and "s" (used in one proof), where I used \tilde
% - for script I, R (ideal used in section 3.4) use \wtcI, \wtcR
% - for complex expressions e.g. X\times Y use \widetilde{...}

\newcommand{\wt}[1]{{\mathchoice%
  {\raisebox{1.5ex}{\resizebox{1.7ex}{!}{{}\hphantom{i}\ensuremath{{\sim}}}} \hspace{-1.7ex}{#1}}%
  {\smash{\raisebox{1.5ex}{\resizebox{1.7ex}{!}{{}\hphantom{i}\ensuremath{{\sim}}}}\hspace{-1.7ex}{#1 }}\vphantom{\tilde I}}%
  {\raisebox{1.1ex}{\resizebox{1.3ex}{!}{{}\hphantom{i}\ensuremath{{\sim}}}}\hspace{-1.3ex}{#1}}%
  {\raisebox{0.8ex}{\resizebox{1ex}{!}{{}\hphantom{i}\ensuremath{{\sim}}}}\hspace{-1ex}{#1}}%
}}

\newcommand{\swt}[1]{{\mathchoice%
  {\raisebox{0.9ex}{\resizebox{1.2ex}{!}{\ensuremath{{\sim}}}}\hspace{-1.4ex}{#1}}%
  {\smash{\raisebox{0.9ex}{\resizebox{1.2ex}{!}{\ensuremath{{\sim}}}}\hspace{-1.4ex}{#1 }}\vphantom{I}}%
  {\raisebox{0.7ex}{\resizebox{0.8ex}{!}{\ensuremath{{\sim}}}}\hspace{-0.9ex}{#1}}%
  {\raisebox{0.5ex}{\resizebox{1ex}{!}{{}\hphantom{i}\ensuremath{{\sim}}}}\hspace{-1ex}{#1}}%
}}

\newcommand{\wtcI}{\smash{\raisebox{1.5ex}{\hspace{0.7ex}\resizebox{1.2ex}{!}{\ensuremath{{\sim}}}}\hspace{-2.1ex}{\cI}}\vphantom{I}}

\newcommand{\wtcR}{\smash{\raisebox{1.5ex}{\hspace{0.7ex}\resizebox{1.2ex}{!}{\ensuremath{{\sim}}}}\hspace{-2.1ex}{\cR}}\vphantom{I}}